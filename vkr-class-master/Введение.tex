\section*{ВВЕДЕНИЕ}
\addcontentsline{toc}{section}{ВВЕДЕНИЕ}

В мире современных компьютерных технологий разработка компьютерных игр остается одним из наиболее захватывающих и творческих направлений в программировании. В этом контексте, жанр 2D платформеров привлекает внимание разработчиков и геймеров своей простотой, ностальгическим воспоминанием о классических играх и бесконечными возможностями для креативного дизайна.

2D платформеры, наследники золотой эры видеоигр, предоставляют уникальный опыт игрокам, основанный на преодолении препятствий, управлении персонажем в ограниченном двухмерном пространстве и погружении в захватывающие приключения. Этот жанр не только оживляет воспоминания о первых видеоиграх, но и постоянно эволюционирует, внедряя новые технологии и идеи в свой дизайн.

Целью данной курсовой работы является исследование и разработка 2D платформера с использованием современных инструментов и технологий. В процессе работы мы сосредоточим внимание на ключевых аспектах геймдизайна, анимации, управления и создания увлекательных уровней. Кроме того, мы рассмотрим технические аспекты, такие как выбор языка программирования, использование специализированных библиотек и создание интерфейса пользователя.

Эта курсовая работа призвана предоставить комплексный обзор процесса разработки 2D платформера, обогатив тем самым понимание основных принципов геймдизайна и программирования в контексте игр данного жанра. В дальнейшем, результаты данной работы могут служить основой для дальнейших исследований и проектов в области разработки компьютерных игр.

\emph{Цель настоящей работы} – разработка игры платформера и её движка. Для достижения поставленной цели необходимо решить \emph{следующие задачи:}
\begin{itemize}
\item провести анализ предметной области;
\item разработать концептуальную модель игры;
\item спроектировать игру;
\item реализовать игру.
\end{itemize}

\emph{Структура и объем работы.} Отчет состоит из введения, 4 разделов основной части, заключения, списка использованных источников, 2 приложений. Текст выпускной квалификационной работы равен \formbytotal{page}{страниц}{е}{ам}{ам}.

\emph{Во введении} сформулирована цель работы, поставлены задачи разработки, описана структура работы, приведено краткое содержание каждого из разделов.

\emph{В первом разделе} на стадии описания технической характеристики предметной области приводится сбор информации.

\emph{Во втором разделе} на стадии технического задания приводятся требования к разрабатываемой игре.

\emph{В третьем разделе} на стадии технического проектирования представлены проектные решения для игры.

\emph{В четвертом разделе} приводится список классов и их методов, использованных при разработке игры, производится тестирование разработанной игры.

В заключении излагаются основные результаты работы, полученные в ходе разработки.

В приложении А представлен графический материал.
В приложении Б представлены фрагменты исходного кода. 
