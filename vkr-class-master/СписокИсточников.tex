\addcontentsline{toc}{section}{СПИСОК ИСПОЛЬЗОВАННЫХ ИСТОЧНИКОВ}

\begin{thebibliography}{9}

    \bibitem{python} Лутц, М. Изучаем Python. Программирование игр, веб-приложений и науки о данных / М. Лутц. – Москва : ДМК Пресс, 2019. – 1280 с. – ISBN 978-5-97060-704-9. – Текст~: непосредственный.
    \bibitem{python} Злобин, С. Программирование игр на Python 3 и Pygame / С. Злобин. – Москва : ДМК Пресс, 2018. – 256 с. – ISBN 978-5-97060-635-6. – Текст~: непосредственный.
    \bibitem{python} Макгроу-Хилл, Э. Python. Эффективное программирование / Э. Макгроу-Хилл. – Санкт-Петербург : Питер, 2018. – 464 с. – ISBN 978-5-496-02911-8. – Текст~: непосредственный.
    \bibitem{python}	Аллен Б. Дауни, Крис Мэйер. Python. Тестирование и разработка / А. Б. Дауни, К. Мэйер. – Санкт-Петербург : Питер, 2020. – 704 с. – ISBN 978-5-4461-1387-3. – Текст~: непосредственный.
	\bibitem{python}	Кэрол А. Грэм. Python. Самоучитель нового поколения / К. А. Грэм. – Москва : Вильямс, 2020. – 480 с. – ISBN 978-5-8459-2146-3. – Текст~: непосредственный.
	\bibitem{python}	Билл Любанович. Python. Тестирование программ. Подробное руководство / Б. Любанович. – Санкт-Петербург : Питер, 2018. – 656 с. – ISBN 978-5-4461-0958-6. – Текст~: непосредственный.
	\bibitem{python}	Аллен Дауни, Эрик Фриман. Python. Программирование на каждый день / А. Дауни, Э. Фриман. – Санкт-Петербург : Питер, 2020. – 528 с. – ISBN 978-5-4461-1352-1. – Текст~: непосредственный.
	\bibitem{python}	Лиза Рассел. Python для детей. Занимательное введение в программирование / Л. Рассел. – Санкт-Петербург : Питер, 2018. – 352 с. – ISBN 978-5-496-03159-3. – Текст~: непосредственный. 
	\bibitem{python}	Лутц, М. Python. Карманный справочник / М. Лутц. – Санкт-Петербург : Питер, 2018. – 416 с. – ISBN 978-5-4461-0926-5. – Текст~: непосредственный.
	\bibitem{python}	Седжвик, Р. Python. Подробное руководство / Р. Седжвик. – Москва : Вильямс, 2017. – 1680 с. – ISBN 978-5-9909015-8-8. – Текст~: непосредственный.
\end{thebibliography}
