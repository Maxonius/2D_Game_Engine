\section{Анализ предметной области}
\subsection{Исследование понятия и история появления жанра платформер}

Жанр компьютерных игр “платформер” (от англ. platform - платформа) зародился в начале 80-х годов прошлого века и стал одним из первых видов видеоигр. Он был создан в результате эволюции аркадных игр, таких как “Space Invaders” и “Pac-Man”, где игрок управлял персонажем на двухмерной плоскости. Платформеры были популярны на игровых автоматах, а затем перешли на домашние компьютеры и игровые консоли.

Платформер - это жанр, в котором игрок управляет персонажем, перемещающимся по платформам и преодолевающим различные препятствия. Главная задача игрока - пройти уровень от начала до конца, собирая различные предметы и избегая опасностей.

Одним из первых представителей жанра считается игра “Donkey Kong” для аркадного автомата Nintendo, выпущенная в 1981 году. В этой игре игрок управлял человеком, пытающимся спасти свою подружку от обезьяны. Персонаж мог прыгать, а также забираться на платформы и ящики. Игра стала очень популярной, и ее успех привел к созданию многих других платформеров на протяжении следующих десятилетий.

В 1982 году компания Nintendo выпустила игру “Mario Bros.”, в которой игрок управлял братом Марио - Луиджи. Эта игра стала основой для многих будущих платформеров, включая серию “Super Mario Bros.”

Ранние платформеры обычно состояли из уровней, где персонаж должен был преодолеть различные препятствия, такие как пропасти, шипы, враги и ловушки, чтобы достичь конца уровня. Некоторые игры также включали элементы головоломки, где игрок должен был использовать предметы или решать задачи, чтобы продвинуться дальше.

С развитием технологий и графики в середине 90-х жанр платформера стал более разнообразным и детализированным. Появились игры с трехмерной графикой, такие как “Crash Bandicoot” и “Spyro the Dragon”, а также игры с элементами паркура, как “Prince of Persia”.

Сегодня платформеры продолжают оставаться популярными, особенно среди детей и подростков. Они часто включают в себя элементы приключений, головоломок и мультиплеера, а также имеют разнообразные сюжетные линии и персонажей. Многие современные игры сочетают элементы платформера с другими жанрами, такими как гонки, шутеры и ролевые игры, создавая новые и интересные игровые опыты.

Жанр платформера продолжает развиваться и адаптироваться к новым технологиям, оставаясь одним из самых популярных и любимых жанров компьютерных игр.
\subsection{Анализ поджанров платформеров}

Жанр платформера включает в себя множество поджанров, каждый из которых имеет свои уникальные особенности и характеристики. Вот некоторые из наиболее распространенных поджанров платформеров:

2D-платформеры: Это классический поджанр платформеров, который включает в себя игры с двухмерной графикой и управлением. Примеры таких игр включают “Super Mario Bros.”, “Kirby’s Epic Yarn” и “Shantae”.

3D-платформеры: Этот поджанр включает в себя платформеры с трехмерной графикой и физикой. Примеры включают “Ratchet and Clank”, “Jak and Daxter” и “Tomb Raider”.

Метроидвания: Это платформеры, которые сочетают в себе элементы метроидвании и платформера. Примеры включают “Castlevania: Symphony of the Night”, “Bloodstained: Ritual of the Night” и “Dead Cells”.

Паркур-платформеры: Эти игры включают в себя элементы паркура и платформера, такие как прыжки, лазание и бег по стенам. Примеры включают серию игр “Mirror’s Edge”, “Assassin’s Creed” и “Prince of Persia”.
\subsection{Анализ существующих разработок}

Среди популярных платформеров можно выделить следующие игры:

Super Mario Bros. - классический 2D платформер, в котором игрок управляет персонажем по имени Марио, преодолевая различные препятствия и сражаясь с врагами.

Kirby’s Epic Yarn - 2D платформер с уникальной графикой и игровым процессом, где игрок управляет персонажем Кирби, способным менять свою форму и использовать различные способности.

Ratchet and Clank - 3D платформер с элементами экшена и головоломок, в котором игроки управляют двумя героями - Рэтчетом и Кланком, сражающимися с различными врагами и решающими загадки.

Castlevania: Symphony of the Night - игра в жанре метроидвания, сочетающая в себе платформер и исследование мира, где игроки управляют вампиром, способным превращаться в различных существ и использовать разное оружие.

Mirror’s Edge - паркур-платформер с видом от первого лица, где игроки могут бегать, прыгать и выполнять различные трюки, преодолевая препятствия и сражаясь с противниками.

Каждая из этих игр имеет свои особенности и уникальный геймплей, который привлекает игроков разных возрастов и предпочтений.
